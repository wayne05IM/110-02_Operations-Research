\documentclass{article}

% Language setting
% Replace `english' with e.g. `spanish' to change the document language
\usepackage[english]{babel}

% Set page size and margins
% Replace `letterpaper' with `a4paper' for UK/EU standard size
\usepackage[letterpaper,top=2cm,bottom=2cm,left=3cm,right=3cm,marginparwidth=1.75cm]{geometry}

% Useful packages
\usepackage{amsmath}
\usepackage{graphicx}
\usepackage[colorlinks=true, allcolors=blue]{hyperref}
\usepackage{tikz}
\usepackage{listings}
\usepackage{tabularx}
\usepackage{mathtools}

\title{Operations Research, Homework 2}
\author{B09705039 LIU, WEI-EN}
\date{}

\begin{document}
\maketitle

\section{Problem 1}

\subsection{(a)}

Standard Form:
\begin{equation}\label{eq:LP}\begin{split}
	\min \quad & x_1 + 2x_2 + 3x_3\\
	\mbox{s.t.} \quad 
	& x_1 + x_2 - x_4 = 4\\
	& x_1 + x_2 + x_3 + x_5 = 9\\
	& x_3 - x_6 = 3\\
	& x_i \geq 0 \quad \forall i \in \{1,......, 6\}\\
\end{split}\end{equation}

\subsection{(b)}

\begin{table}[htbp]
\centering
\begin{tabularx}{1\textwidth}
{ | >{\centering\arraybackslash}X 
  | >{\centering\arraybackslash}X 
  | >{\centering\arraybackslash}X |} \hline
Basis & Bfs & Basic solutions \\\hline
($x_1, x_3, x_4$) & Yes & $(6, 0, 3, 2, 0, 0)$ \\\hline
($x_1, x_3, x_5$) & Yes & $(4, 0, 3, 0, 2, 0)$ \\\hline
($x_1, x_3, x_6$) & Yes & $(4, 0, 5, 0, 0, 2)$ \\\hline
($x_1, x_4, x_6$) & No & $(9, 0, 0, 5, 0, -3)$ \\\hline
($x_1, x_5, x_6$) & No & $(4, 0, 0, 0, 5, -3)$ \\\hline
($x_2, x_3, x_4$) & Yes & $(0, 6, 3, 2, 0, 0)$ \\\hline
($x_2, x_3, x_5$) & Yes & $(0, 4, 3, 0, 2, 0)$ \\\hline
($x_2, x_3, x_6$) & Yes & $(0, 4, 5, 0, 0, 2)$ \\\hline
($x_2, x_4, x_6$) & No & $(0, 9, 0, 5, 0, -3)$ \\\hline
($x_2, x_5, x_6$) & No & $(0, 4, 0, 0, 5, -3)$ \\\hline
($x_3, x_4, x_5$) & No & $(0, 0, 3, -4, 6, 0)$ \\\hline
($x_3, x_4, x_6$) & No & $(0, 0, 9, -4, 0, 6)$ \\\hline
($x_4, x_5, x_6$) & No & $(0, 0, 0, -4, 9, -3)$ \\\hline
\end{tabularx}
\caption{\label{tab:widgets}Table of basic solutions $(x_1, x_2, x_3, x_4, x_5, x_6)$ and basic feasible solutions(Bfs).}
\end{table}

\subsection{(c)}

6 extreme points ($x_1, x_2, x_3$):\\

(6, 0, 3), (4, 0, 3), (4, 0, 5), (0, 6, 3), (0, 4, 3), (0, 4, 5)

\newpage
\subsection{(d)}
Phase 1 LP:\\

\begin{equation}\label{eq:LP}\begin{split}
	\min \quad & x_7 + x_8\\
	\mbox{s.t.} \quad 
	& x_1 + x_2 - x_4 + x_7 = 4\\
	& x_1 + x_2 + x_3 + x_5 = 9\\
	& x_3 - x_6 + x_8 = 3\\
	& x_i \geq 0 \quad \forall i \in \{1,......, 8\}\\
\end{split}\end{equation}

\begin{tabular}{cccccccc|c}
0 & 0 & 0 & 0 & 0 & 0 & -1 & -1 & 0\\
\hline
1 & 1 & 0 & -1 & 0 & 0 & 1 & 0 & $x_7 = 4$\\
1 & 1 & 1 & 0 & 1 & 0 & 0 & 0 & $x_5 = 9$\\
0 & 0 & 1 & 0 & 0 & -1 & 0 & 1 & $x_8 = 3$\\
\end{tabular}
$\xrightarrow{\text{adjust}}$
\begin{tabular}{cccccccc|c}
1 & 1 & 1 & -1 & 0 & -1 & 0 & 0 & 7\\
\hline
\fbox{1} & 1 & 0 & -1 & 0 & 0 & 1 & 0 & $x_7 = 4$\\
1 & 1 & 1 & 0 & 1 & 0 & 0 & 0 & $x_5 = 9$\\
0 & 0 & 1 & 0 & 0 & -1 & 0 & 1 & $x_8 = 3$\\
\end{tabular}
\\\\

$\xrightarrow{}$
\begin{tabular}{ccccccc|c}
0 & 0 & 1 & 0 & 0 & -1 & 0 & 3\\
\hline
1 & 1 & 0 & -1 & 0 & 0 & 0 & $x_1 = 4$\\
0 & 0 & 1 & -1 & 1 & 0 & 0 & $x_5 = 5$\\
0 & 0 & \fbox{1} & 0 & 0 & -1 & 1 & $x_8 = 3$\\
\end{tabular}
$\xrightarrow{}$
\begin{tabular}{cccccc|c}
0 & 0 & 0 & 0 & 0 & 0 & 0\\
\hline
1 & 1 & 0 & -1 & 0 & 0 & $x_1 = 4$\\
0 & 0 & 0 & -1 & 1 & 1 & $x_5 = 2$\\
0 & 0 & 1 & 0 & 0 & -1 & $x_3 = 3$\\
\end{tabular}
\\\\
\\Phase 2 LP:\\

\begin{tabular}{cccccc|c}
-1 & -2 & -3 & 0 & 0 & 0 & 0\\
\hline
1 & 1 & 0 & -1 & 0 & 0 & $x_1 = 4$\\
0 & 0 & 0 & -1 & 1 & 1 & $x_5 = 2$\\
0 & 0 & 1 & 0 & 0 & -1 & $x_3 = 3$\\
\end{tabular}
$\xrightarrow{\text{adjust}}$
\begin{tabular}{cccccc|c}
0 & -1 & -3 & -1 & 0 & 0 & 4\\
\hline
1 & 1 & 0 & -1 & 0 & 0 & $x_1 = 4$\\
0 & 0 & 0 & -1 & 1 & 1 & $x_5 = 2$\\
0 & 0 & 1 & 0 & 0 & -1 & $x_3 = 3$\\
\end{tabular}
\\\\

$\xrightarrow{\text{adjust}}$
\begin{tabular}{cccccc|c}
0 & -1 & 0 & -1 & 0 & -3 & 13\\
\hline
1 & 1 & 0 & -1 & 0 & 0 & $x_1 = 4$\\
0 & 0 & 0 & -1 & 1 & 1 & $x_5 = 2$\\
0 & 0 & 1 & 0 & 0 & -1 & $x_3 = 3$\\
\end{tabular}
\\\\
\\The standard form LP:\\
Optimal solution = $(x_1, x_2, x_3, x_4, x_5, x_6)$ = (4, 0, 3, 0, 2, 0)\\
Objective value = 13\\
\\
\\The original LP:\\
Optimal solution = $(x_1, x_2, x_3)$ = (4, 0, 3)\\
Objective value = 13\\
\\
\\In the minimization problem an iteration has no improvement when all the coefficient of it's objective function is $\leq 0$. When this situation happens, we skip these iterations. Thus, there isn't any iteration presented above that has no improvements.

\newpage
\section{Problem 2}

\subsection{(a)}

Linear Relaxation:
\begin{equation}\label{eq:LP}\begin{split}
	\max \quad & 2x_1 + 2x_2 + 5x_3 + 11x_4 + 10x_5\\
	\mbox{s.t.} \quad 
	& x_1 + 4x_2 + 3x_3 + 5x_4 + 3x_5 \leq 20\\
	& x_i \geq 0 \quad \forall i \in \{1,......, 5\}\\
\end{split}\end{equation}
	
Ratio for greedy(objective function coefficient / constraint 1 coefficient): $x_5 > x_4 > x_1 > x_3 > x_2$\\

LR optimal solution: $(x_1, x_2, x_3, x_4, x_5) = (0, 0, 0, 0, {\frac{20}{3}})$\\

objective value = ${\frac{200}{3}}$

\subsection{(b)}

Branch-and-bound tree:

\begin{center}
\begin{tikzpicture}
    [sibling distance=10em,level distance=6em,
     every node/.style={shape=rectangle,draw,align=center}]
\centering
\node{$x = (0, 0, 0, 0, {\frac{20}{3}})$ \\ $z = {\frac{200}{3}}$}
    child{node{$x = (0, 0, 0, {\frac{2}{5}}, 6)$ \\ $z = {\frac{322}{5}}$}
        child{node{$x = (2, 0, 0, 0, 6)$ \\ $z = 64$} edge from parent node[left,draw=none] {$x_4 \leq 0$}}
        child{node{INF} edge from parent node[right,draw=none] {$x_4 \geq 1$}} edge from parent node[left,draw=none] {$x_5 \leq 6$}}
    child{node{INF} edge from parent node[right,draw=none] {$x_5 \geq 7$}}
\end{tikzpicture}
\end{center}

optimal solution $(x_1, x_2, x_3, x_4, x_5)$ = $(2, 0, 0, 0, 6)$\\

objective value = 64

\newpage
\section{Problem 3}

\subsection{(a)}
Ranges:

$N = \{1,\ldots, n\}$ as range of jobs.

$M = \{1,\ldots, m\}$ as range of machines.\\
\\
Coefficients:

$p_j$ is the processing time(hr) of job $j$, $j \in J$.

$b_j$ is the benefit of job $j$, $j \in J$.\\
\\
Variables:

$y$ represent the total benefit of the least total benefit machine.

$x_{mj}$ is $1$ if job $j$ is scheduled to machine m, $0$ otherwise.

\begin{equation}\label{eq:IP}\begin{split}
	\max \quad & y\\
	\mbox{s.t.} \quad 
	& y \leq \sum_{j \in N} x_{mj}b_j \quad \forall m \in M\\
	& \sum_{m \in M} x_{mj} \leq 1 \quad \forall j \in N\\
	& \sum_{j in N}x_{mj}p_j \leq K \quad \forall m \in M\\
	& x_{mj} = \{0, 1\} \quad \forall m \in M, j \in N\\
\end{split}\end{equation}

\subsection{(b)}

Linear relaxation:

$x_{mj}$ is the proportion of job $j$ scheduled to machine m.

Others are same as (a).

\begin{equation}\label{eq:IP}\begin{split}
	\max \quad & y\\
	\mbox{s.t.} \quad 
	& y \leq \sum_{j \in N} x_{mj}b_j \quad \forall m \in M\\
	& \sum_{m \in M} x_{mj} \leq 1 \quad \forall j \in N\\
	& \sum_{j in N}x_{mj}p_j \leq K \quad \forall m \in M\\
	& x_{mj} = [0, 1] \quad \forall m \in M, j \in N\\
\end{split}\end{equation}
\\
\\The linear relaxation optimal solution is to schedule the whole job $1, 3$ to machine 1, job $7, 10$ to machine 2, job $4, 5$ to machine 3, do not schedule job 9. Split job 2 into $(0.8333, 0.119, 0.0476)$, job 6 into $(0.25, 0.0, 0.75)$, job 8 into $(0.0, 0.2738, 0.3095)$ and schedule them into machine 1 to 3 respectively.
\\
\\The objective value is $17.416666666666668$.
\\
\\It is an upper bound of the objective value of an IP-optimal solution, because it maximizes the objective function and according to the proposition discussed in class the linear relaxation objective value is lager than a maximization IP objective value.
\\
\\Python Code:
\begin{lstlisting}[breaklines = true, language=Python]
# Import libraries
from gurobipy import *
import pandas as pd

# Input
m = 3
n = 10
b = [5, 8, 4, 6, 3, 7, 6, 9, 5, 8]
p = [3, 6, 5, 4, 1, 8, 5, 12, 7, 6]
K = 15

N = range(n)
M = range(m)

# Modeling
# add var
p3_1 = Model("problem3-1")    # build a new model

x = p3_1.addVars(M, N, lb = 0, ub = 1, vtype = GRB.CONTINUOUS, name = "x")
y = p3_1.addVar(lb = 0, vtype = GRB.CONTINUOUS, name = "y")

# setting the objective function 
p3_1.setObjective(
    y
    , GRB.MAXIMIZE) 

# add constraints and name them
for i in M:
    p3_1.addConstr(y <= (quicksum(x[i, j] * b[j] for j in N)), name = f"min benefit")
    
for j in N:
    p3_1.addConstr(quicksum(x[i, j] for i in M) <= 1, name = f"distribute to no more than 1 machine")
    
for i in M:
    p3_1.addConstr(quicksum(x[i, j] * p[j] for j in N) <= K, name = f"satisfy capacity")
    
p3_1.optimize()

# Results
print("Results:\n")

# objective value
LR_ov = p3_1.objVal
print("objective value =", LR_ov)
print("")
    
# x
print("x:")
print("J \ M", end="")
print("\tMachine1 \tMachine2 \tMachine3")
for j in N:
    print("Job" + str(j+1), "\t", end="")
    for i in M:
        print(round((x[i, j].x), 4), "\t\t", end="")
    print("")
print("")

# y
print("y:")
print(y.x)
\end{lstlisting}

\subsection{(c)}

The optimal solution is to schedule job $10, 2$ to machine 1, job $20, 11$ to machine 2, job $3, 14, 15$ to machine 3, job $19, 25, 6$ to machine 4 and don't schedule the rest of the jobs.
\\
\\Then we will get the optimal objective value (machine earning the least benefit) $23$.
\\
\\Machine 1 earns 23, machine 2 earns 23, machine 3 earns 23, machine 4 earns 28 benefits.
\\
\\The percentage optimality gap with respect to the bound obtained by linear relaxation is $0.2432432432432433$ (about $24.3\%$).
\\
\\Python Code:
\begin{lstlisting}[breaklines = true, language=Python]
# Sorting fuctions
def B_bubbleSort(b, job):
    for i in N:
        for j in range(0, n-i-1):
            if b[job[j]] < b[job[j+1]]:
                job[j], job[j+1] = job[j+1], job[j]
                
def M_bubbleSort(bm, machine):
    for i in M:
        for j in range(0, m-i-1):
            if bm[machine[j]] > bm[machine[j+1]]:
                machine[j], machine[j+1] = machine[j+1], machine[j]
                
# Input
m = 4
n = 25
b = [5,8,12,4,6,6,3,7,6,15,9,5,8,10,1,5,3,7,12,14,5,8,9,8,10]
p = [3,6,7,5,4,2,6,3,5,8,10,2,4,7,1,5,8,3,6,4,12,4,8,4,7]
K = 15

N = range(n)
M = range(m)

# Solve
# Step 1: Sorting
job_sorted = []
for i in N:
    job_sorted.append(i)

B_bubbleSort(b, job_sorted)
    
# Step 2: Scheduling
job_to_machine = [0] * n
Pm = [0] * m
Bm = [0] * m

for i in job_sorted:
    # sort machine in priority
    machine_sorted = []
    for k in M:
        machine_sorted.append(k)
    M_bubbleSort(Bm, machine_sorted)
    
    # schedule jods to machine
    scheduled = False
    for j in machine_sorted:
        if (Pm[j] + p[i] <= K):
            job_to_machine[i] = j
            Pm[j] += p[i]
            Bm[j] += b[i]
            scheduled = True
            break         
    if (scheduled == False):
        job_to_machine[i] = -1
        
# print results (machine number = 0 in the display part if it is not scheduled)
for i in job_sorted:
    print("job", i + 1, ", machine", job_to_machine[i] + 1)
display("Accumulated processing time:", Pm)
display("Accumulated benefits:", Bm)
HBF_ov = min(Bm)
print("objective value:", HBF_ov)
print("HBF optimality gap:", (LR_ov - HBF_ov) / LR_ov)
\end{lstlisting}

\subsection{(d)}

The optimal solution is to schedule job $20, 24, 3$ to machine 1, job $6, 13, 10$ to machine 2, job $12, 18, 22, 5$ to machine 3, job $8, 19, 1, 15$ to machine 4 and don't schedule the rest of the jobs.
\\
\\Then we will get the optimal objective value (machine earning the least benefit) $25$.
\\
\\Machine 1 earns 34, machine 2 earns 29, machine 3 earns 26, machine 4 earns 25 benefits.
\\
\\The percentage optimality gap with respect to the bound obtained by linear relaxation is $0.17743830787309056$ (about $17.7\%$).
\\
\\Python Code:
\begin{lstlisting}[breaklines = true, language=Python]
# Sorting fuctions
def R_bubbleSort(b, job):
    for i in N:
        for j in range(0, n-i-1):
            if (b[job[j]]/p[job[j]]) < (b[job[j+1]]/p[job[j+1]]):
                job[j], job[j+1] = job[j+1], job[j]
                
def M_bubbleSort(bm, machine):
    for i in M:
        for j in range(0, m-i-1):
            if bm[machine[j]] > bm[machine[j+1]]:
                machine[j], machine[j+1] = machine[j+1], machine[j]

# Solve
# Step 1: Sorting
job_sorted = []
for i in N:
    job_sorted.append(i)

R_bubbleSort(b, job_sorted)
    
# Step 2: Scheduling
job_to_machine = [0] * n
Pm = [0] * m
Bm = [0] * m

for i in job_sorted:
    # sort machine in priority
    machine_sorted = []
    for k in M:
        machine_sorted.append(k)
    M_bubbleSort(Bm, machine_sorted)
    
    # schedule jods to machine
    scheduled = False
    for j in machine_sorted:
        if (Pm[j] + p[i] <= K):
            job_to_machine[i] = j
            Pm[j] += p[i]
            Bm[j] += b[i]
            scheduled = True
            break         
    if (scheduled == False):
        job_to_machine[i] = -1
        
# print results (machine number = 0 in the display part if it is not scheduled)
for i in job_sorted:
    print("job", i + 1, ", machine", job_to_machine[i] + 1)
display("Accumulated processing time:", Pm)
display("Accumulated benefits:", Bm)
HRF_ov = min(Bm)
print("objective value:", HRF_ov)
print("HRF optimality gap:", (LR_ov - HRF_ov) / LR_ov)
\end{lstlisting}

\newpage
\section{Problem 4}

\subsection{(a)}

\begin{table}[htbp]
\centering
\begin{tabularx}{1\textwidth}
{ | >{\centering\arraybackslash}X 
  | >{\centering\arraybackslash}X 
  | >{\centering\arraybackslash}X
  | >{\centering\arraybackslash}X |} \hline
Instance & HBF & HRF & Self define(HRBF)\\\hline
1 & 0.0184 & 0.0184 & 0.0116 \\\hline
2 & 0.0659 & 0.0344 & 0.0344 \\\hline
3 & 0.0863 & 0.068 & 0.068 \\\hline
4 & 0.058 & 0.0303 & 0.0303 \\\hline
5 & 0.1809 & 0.0806 & 0.0806 \\\hline
6 & 0.0655 & 0.0396 & 0.0309 \\\hline
7 & 0.0949 & 0.0353 & 0.0472 \\\hline
8 & 0.0859 & 0.0728 & 0.0728 \\\hline
9 & 0.0671 & 0.0671 & 0.0671 \\\hline
10 & 0.0101 & 0.0173 & 0.0173 \\\hline
11 & 0.1718 & 0.0535 & 0.0535 \\\hline
12 & 0.0565 & 0.0307 & 0.0242 \\\hline
13 & 0.0756 & 0.1092 & 0.1261 \\\hline
14 & 0.1069 & 0.0611 & 0.0611 \\\hline
15 & 0.0686 & 0.0599 & 0.0427 \\\hline
16 & 0.0583 & 0.0743 & 0.0743 \\\hline
17 & 0.0116 & 0.0665 & 0.0665 \\\hline
18 & 0.1332 & 0.0648 & 0.0648 \\\hline
19 & 0.0387 & 0.1099 & 0.1099 \\\hline
20 & 0.102 & 0.0694 & 0.0694 \\\hline
average & 0.0778 & 0.0581 & 0.0576 \\\hline
\end{tabularx}
\caption{\label{tab:widgets}Table for optimality gaps of the three algorithms.}
\end{table}

Comment of the performance of the two heuristic algorithms:

According to the table 2 above, the average optimality gap of HBF is $0.0778$, HRF is $0.0581$, $0.0778 > 0.0581$, which implies that HRF is closer to the bound obtained by linear relaxation. Thus, HRF is a better heuristic algorithm compared with HBF.

\subsection{(b)}
(1) Inspired by the above two algorithms, I had an idea of gaining advantages of each algorithm by combining both. Since HRF is better than HBF in the previous problem, I started to sort the jobs by ratio first, and then sort jobs that have the same ratio with benefit first. By combining the benefits of both methods, I expected this self defined algorithm to have a slightly better solution than the original HRF algorithm.\\
\\(2) The results are in the last column (Self define(HRBF)) in table 2 above. It performed as I expected, the average optimality gap is slightly better than the HRF algorithm and it is the best among all heuristic algorithm in the table.
\\
\\Python Code:
\begin{lstlisting}[breaklines = true, language=Python]
# Sorting fuctions
def RB_bubbleSort(b, job):
    for i in N:
        for j in range(0, n-i-1):
            if (b[job[j]]/p[job[j]]) < (b[job[j+1]]/p[job[j+1]]):
                job[j], job[j+1] = job[j+1], job[j]        
    for i in N:
        for j in range(0, n-i-1):
            if (b[job[j]]/p[job[j]]) == (b[job[j+1]]/p[job[j+1]]) and (b[job[j]] < b[job[j+1]]):
                job[j], job[j+1] = job[j+1], job[j]
                
def M_bubbleSort(bm, machine):
    for i in M:
        for j in range(0, m-i-1):
            if bm[machine[j]] > bm[machine[j+1]]:
                machine[j], machine[j+1] = machine[j+1], machine[j]
                
# file input ('in01.txt' for example)
f = open('in01.txt', 'r')
line1 = f.readline()
line2 = f.readline()
line3 = f.readline()

array = line1.split( )
array = list(map(int, array))
    
m = array[0]
n = array[1]
K = array[2]

b = line2.split( )
b = list(map(int, b))

p = line3.split( )
p = list(map(int, p))

N = range(n)
M = range(m)
f.close()

# Solve
# Step 1: Sorting
job_sorted = []
for i in N:
    job_sorted.append(i)

RB_bubbleSort(b, job_sorted)
'''
for i in job_sorted:
    print(i, b[i])
print("")
'''
    
# Step 2: Scheduling
job_to_machine = [0] * n
Pm = [0] * m
Bm = [0] * m

for i in job_sorted:
    # sort machine in priority
    machine_sorted = []
    for k in M:
        machine_sorted.append(k)
    M_bubbleSort(Bm, machine_sorted)
    
    # schedule jods to machine
    scheduled = False
    for j in machine_sorted:
        if (Pm[j] + p[i] <= K):
            job_to_machine[i] = j
            Pm[j] += p[i]
            Bm[j] += b[i]
            scheduled = True
            break         
    if (scheduled == False):
        job_to_machine[i] = -1
        
# print results (machine number = 0 in the display part if it is not scheduled)
for i in job_sorted:
    print("job", i + 1, ", machine", job_to_machine[i] + 1)
display("Accumulated processing time:", Pm)
display("Accumulated benefits:", Bm)
HRBF_ov = min(Bm)
print("objective value:", HRBF_ov)
print("HRBF optimality gap:", (LR_ov - HRBF_ov) / LR_ov)
\end{lstlisting}

\newpage
\section{Problem 5}
$i_k = (x_1, x_2)$ of iteration $k$.
\\
\\$f(i_k) = 3x_1^2 + 2x_2^2 +  4 x_1 x_2 + 6e^{x_1} + x_2$
\\
\\Gradient(G($i_k$)) =
$$\begin{bmatrix}
6x_1 + 4x_2 + 6e^{x_1}\\
4x_2 + 4x_1 + 1
\end{bmatrix}$$
\\
\\Hessian matrix(H($i_k$)) =
$$\begin{bmatrix}
6 + 6e^{x_1} & 4\\
4 & 4
\end{bmatrix}$$

\subsection{(a)}
Iteration 0:
\begin{equation*}
i_0 = (x_1, x_2) = (0, 0), f(i_0) = 6, G(i_0) = 
\begin{bmatrix}
6 \\
1
\end{bmatrix}
\end{equation*}
\\
\\Iteration 1:
\begin{equation*}
a_0 = argmin_{a \geq 0}f(i_0 - aG(i_0)) = argmin_{a \geq 0}134a^2 - a + 6e^{-6a}\\
\xRightarrow{} a = 0.0845924274342, a_0 = 4.48609
\end{equation*}
\\$i_1$ = (-0.5075545646052, -0.0845924274342), objective value = 4.48609

\subsection{(b)}
Iteration 0:
\begin{equation*}
i_0 = (x_1, x_2) = (0, 0), G(i_0) = 
\begin{bmatrix}
6 \\
1
\end{bmatrix}
, H(i_0) = 
\begin{bmatrix}
12 & 4 \\
4 & 4
\end{bmatrix}
\end{equation*}
\\
\\Iteration 1:
\begin{equation*}
G(i_0) + H(i_0)(i_1 - i_0) = 0 \xRightarrow{} i_1 = i_0 - [H(i_0)]^{-1}G(i_0)
\end{equation*}
\begin{equation*}
\xRightarrow{} i_1 =
\begin{bmatrix}
0 \\
0
\end{bmatrix}
- {\frac{1}{8}}
\begin{bmatrix}
1 & -1\\
-1 & 3
\end{bmatrix}
\begin{bmatrix}
6\\
1
\end{bmatrix}
=
\begin{bmatrix}
-{\frac{5}{8}}\\
{\frac{3}{8}}
\end{bmatrix}
\end{equation*}
\\$i_1$ = ($-{\frac{5}{8}}$, ${\frac{3}{8}}$), objective value = 4.102193571113942

\end{document}